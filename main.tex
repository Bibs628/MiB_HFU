% ================================================================================================ %
% Hilfe zum Einstieg in LaTeX
% ================================================================================================ %

% Hier sind ein paar nützliche Links um LaTeX besser lernen zu können und ein paar die es sich zu merken einfach lohnt.

% Ein guter Einstieg ist die Overleaf Dokumentation, diese nimmt dich gut an die Hand.
    % https://de.overleaf.com/learn

% Eine weitere gute Quelle ist "The LaTeX Project", diese bieten einen weiteren gute Dokumentation (auch in De) um hier einen umfassenden Einstig zu finden.
    % https://www.latex-project.org/help/documentation/

% Wenn man schnell LaTeX lernen will empfehle ich die 5 Artikel von Overleaf zum Thesis schreiben
    % https://de.overleaf.com/learn/latex/How_to_Write_a_Thesis_in_LaTeX_(Part_1)%3A_Basic_Structure

% oder die 18 kurze und anschauliche Kapitel von Latex-tutorials.com oder iht Quikstart-Guide
    % https://latex-tutorial.com/tutorials/
    % https://latex-tutorial.com/quick-start/
    
% Auch sinnvoll ist es bei Problemen zu Googeln oder ne KI-Chatbox zu fragen, die meisten können mit LaTeX umgehen und geben aus meiner Erfahrung relativ Sinnvolle antworten

% ======= Resourcen, die Sinnvoll sind %
% Hyperref Anpassungen
    % https://www.namsu.de/Extra/pakete/Hyperref.html
% Biblatex Quide zum einstieg in die Quellen
    % https://de.overleaf.com/learn/latex/Bibliography_management_with_biblatex
% nutzung von Farben in LaTeX (mit xColor)
    % https://latex-tutorial.com/color-latex/
    


\documentclass[             % Basiseinstellungen des Dokuments
12pt,                           % Schriftgröße
a4paper                         % Blattformat
]{report}                       % Art des Textes (Report, Book, Article, ...)
\usepackage[utf8]{inputenc} % Unicode Standart für Sonderzeichen
\usepackage[T1]{fontenc}    % Hinzufügen von mehr Seitenumbrüchen
\usepackage[]{fancyhdr}     % Hinzufügen von mehr Seitenstyles
\pagestyle{fancy}           % Änderung zum "fancy" Seitenstyles
\usepackage[ngerman]{babel} % Deutsches Sprachpaket mit neue Rechtschreibung, d.\,h. (Silbentrennung)
\usepackage[dvipsnames, x11names ]{xcolor} % zusätzliche Farben
\usepackage{multirow}       % Hinzufügen von Tabellen
\usepackage{array}          % Mehr optionen bei Tabellen 
\usepackage{graphicx}       % besseres Hinzufügen und Formatieren von Bildern 
\usepackage[iso, german]{isodate}      % formatieren von Datum, Zeit und Zeitzonen
\usepackage{hyperref}       % Hinzufügen von internen und externen Links
\usepackage{lipsum}         % fügt Text hinzu (Lorem Ipsum)       
  \usepackage[              % fügt Biblatex für vernünftige Quellen ein
    backend=biber,              % Formatierung vom Backend der Quellendokumente
    style=numeric,              % Definiert die Formatierung 
  ]{biblatex}
\usepackage{colortbl}       % Ermöglicht es Tabellen Farbig zu gestalten
\usepackage{movie15}        % Ermöglicht das einbinden von .gif datein durch den befehl
\usepackage{awesomebox}
\usepackage{makecell}       % Ermöglicht linebreaks in Tabellen durch \makecell{} und Überschriften dank \thead{} 
\usepackage{longtable}      % Ermöglicht das Erstellen von Listen über mehreren Seiten
\usepackage{wrapfig}        % Ermöglicht es das Wrappen von Bildern, gut für Bild neben Text 
\usepackage{caption}        % Dient zum erstellen von figures mit mehreren Bildern
\usepackage{subcaption}     % Dient zum erstellen von figures mit mehreren Bildern
% ================================================================================================ %
% Anpassungen & Befehle
% ================================================================================================ %


\hypersetup{                % Einstellung von Hyperref, diese Infos sind allgeimeine Dokumenteinstellungen
% ======= Dokumentoptionen %
pdftoolbar=true,	% Anzeigen der Acrobat toolbar oder nicht
pdfmenubar=true,	% Anzeigen des Acrobat menu oder nicht
pdftitle={Zusammenfassung Semester 1 Fakultät DM},	% Titel
pdfsubject={lernen},	% Um was geht es
pdfauthor={Nils J. Hack, },	% Autor bzw. Autoren
pdfkeywords={Hochschule Furtwangen, Fakultät Digitale Medien, SoSe24},	% Stichwort1, Stichwort2 ...
pdfcreator={Overleaf},	% Mit welcher Anwendung i.d.R. pdflatex
pdfproducer={Text},	% LaTeX with hyperref
% ======= %
bookmarks=true,	            % erstellt Bookmarks
bookmarksopen=true,        % Anzeigen der Bookmarks beim Öffen des Dokuments
bookmarksnumbered=false,	% Anschnittsnummer anzeigen
% ======= %
    colorlinks=true,            % Farblinks an
    linkcolor=YellowGreen,      % Farbe von internen Verweise
    citecolor=Cerulean,         % Farbe von Zitaten
    urlcolor=RedViolet,         % Farbe zu externen Links
    }

% ============= %
\renewcommand{\footrulewidth}{0.4pt}    % 
% ============= %
\setlength\parindent{0pt}
% ============= %
\setcounter{secnumdepth}{5}
% ============= %

% ======= Quellenbibliotheken hinzufügen %
\addbibresource{Quellen/Medientechnik.bib}



% ======= Formatierungen des Seitenlayouts %
\fancyhead[R]{\rightmark}   % Kopfzeile mit aktueller Section rechts
\fancyhead[L]{\leftmark}    % Kopfzeile mit aktuellem Kapitel oben links
\fancyfoot[R]{\hyperref[TOC]{\textcolor{YellowGreen}{Inhaltsverzeichnis}}}  % fügt unten rechts einen link zum Inhaltsverzeichnis hinzu
\fancyfoot[L]{SoSe24 DM Sem1} % fügt unten links den Text hinzu


% ================================================================================================ %
% Dokument beginnt
% ================================================================================================ %

\begin{document}

% ======= Titelseite anpassen %
\vspace{-5cm}
\title{\vspace{-5cm}\begin{center}
    \includegraphics[height=100px]{Bilder/HFU_logo.png}
\end{center}\vspace{3cm}Zusammenfassung\\ Semester 1\\ Fakultät DM}
\author{Zusammengefasst von:\\\emph{Nils Hack \& Daniel Georg}}
\date{Im Zeitraum:\\12. Juli 2024 - \today}

% ======= Titelseite %
\maketitle
\setcounter{page}{2} % Setzt Seitenzahl auf 2 weil Deckblatt geskippt wird
\newpage

% ======= Ínhaltsverzeichnis %
\hypersetup{linkcolor=black}    % Setzt die Farbe des Inhaltsverzeichnis zu schwarz, überschreibt das Seitenlayout
\tableofcontents\label{TOC}     % fügt ein label ein um dieses zu Verlinken
\newpage

% ======= Inhalte %             % Durch das hinzufügen von einem "%" die ungewünchten Bereiche ausklammern
\section{Vorwort}
    \subsection{Formattierung}
Unten Rechts am Rand unter \textcolor{YellowGreen}{Inhaltsverzeichnis} ist ein Link der dich wieder nach oben zum Inhaltsverzeichnis bringt.\\~\\
Zudem wird obenlinks das Kapitel und oben rechts das unterkapitel angezeigt um die Orientierung zu verbessern.\\~\\
Im folgenem Dokument werden \\
\textcolor{YellowGreen}{interne Link} in YellowGreen,   \\   
\textcolor{Cerulean}{Quellenverlinkungen} in SeaGreen und \\         
\textcolor{RedViolet}{externe Links} in  RedViolet dargesellt.
    \subsection{Zum Inhalt}
Dies hier zuammengestelten Seiten versuchen ein Gesamtbild aller Vorlesungen abzubilden, die im 1. Semester in der noch so genannten Fakultät Digitale Medien beigebrachten Inhalte näher zu bringen. Dies wird wahrscheinlich nicht im vollem Umfang möglich sein und ich bitte dies zu berücksichtigen. Auch werden wir Versuchen hier aktuelle Klausuraufgaben möglichst gut wiederzugeben, jedoch halten wir uns hier an dem gesetzlichen Rahmen und werden diese entsprechend nicht 1:1 veröffentlichen.\\~\\
Im inhaltlichen Fangen wir bei der SPO an, um einen Rahmen zu bieten und um darauf inhaltlich Aufbauen zu können. Wir Orientieren uns nach der Bennenung nach der SPO und entsprechend sind die Kapitel getaltet. Am Ende von jedem Kurs wollen wir noch kurz mögliche Klausuraufgaben erklären und wo es Sinn ergibt auch das Inhaltliche etwas erweitern und evtl. praktische Tipps näher zu bringen.\\~\\
\emph{- Nils}
%\chapter{AWBL Maier}

Der Herr Maier ist ein Professor aus der Fakultät Wirtschaft der entsprechend bei uns einspringt. Er macht meiner Meinung nach guten Untericht, auch wenn das Thema entsprechend manchmal schlauchen kann. Es empfiehlt sich entsprechend zu seiner Vorlesungen zu kommen weil er doch für die Prüfung entsprechend wichtige Inhalte vermittelt. Bei ihm sind zumindest auch recht hohe Durchfallquoten recht normal, es liegt aber auch an der zu teilen schlechten Anwesendheit in den Vorlesungen.\\
Falls ihr jedoch Fragen habt könnt ihr gerne auf ihn zu kommen, er wird euch in der Regel diese beantworten. Falls ihr z.B. auf einem Wirtschaftsgymnasium wart oder ähnliche Sachen nachweisen könnt hat er dies in irgend einer Form (weiß es nicht genauer) angerechnet.
\\~\\
\emph{- Nils}

\section{SPO}

    Nachdem Studierende das Modul erfolgreich abgeschlossen haben, können sie:
    \begin{itemize}
        \item Wissen / Kenntnisse\\
            Das Mikro- und Makro-Umfeld von Medienbetrieben, und wie sie diese beeinflussen, benennen sowie den Stellenwert der Medienbranche in der Volkswirtschaft und Gesellschaft skizzieren.
            \newline
            Erklären, wie Medienunternehmen aus betriebswirtschaftlicher Sicht grundlegend funktionieren sowie die relevanten regulatorischen Bedingungen für das Medienmanagement kennen.
        \item Verstehen\\
            Verstehen, wie sich Medienbetrieb unterschiedlicher Art finanzieren sowie verstehen, welche Rechtsformen Medienbetriebe haben können.
            \newline
            Verstehen, welche strategischen und operativen Entscheidungen Medienunternehmen treffen müssen sowie welche Managementinstrumente Medienunternehmen benutzen (können).
        \item Anwenden\\
            Darlegen, in welchem volkswirtschaftlichen, politischen und regulatorischen Bezugsrahmen Medienbetriebe agieren. 
            \newline
            Benennen, wie einzelne Medienbetriebe ihren Markt bzw. ihre Branche definieren und wie sie dies in ihren Aktivitäten beeinflusst.
        \newpage
        \item Analyse\\
            Analysieren, wie Angebot und Nachfrage von Mediengütern zusammenspielen und wie dies von Medienbetrieben koordiniert wird sowie Investitionsentscheidungen in Medienbetrieben analysieren.
            \newline
            Analysieren, wie Medienunternehmen organisiert sind sowie welche Auswirkungen regulatorische Bedingungen auf Entscheidungen im Medienmanagement haben.
        \item Synthesis\\
            Allgemeine personalpolitische Maßnahmen auf Medienbetriebe übertragen.
        \item Evaluation\\
            Steuerliche Konsequenzen medienbetrieblicher Entscheidungen grob bewerten.
            \newline
            Medienbetriebliche Entscheidungen aus Sicht des Controllings bewerten.
    \end{itemize}
    
\newpage
\section{Allgemeine BWL}

\notebox{Aufgabe der Betriebswirtschaftslehre ist es, alles wirtschaftliche Handeln, das sich im Betrieb vollzieht, zu beschreiben und zu erklären und schließlich auf Grund der erkannten Regelmäßigkeiten und Gesetzmäßigkeiten des Betriebsprozesses wirtschaftliche Verfahren zur Realisierung praktischer betrieblicher Zielsetzungen zu entwickeln.}


\newpage\section{Unternehmensgründung}
    \subsection{Grundlagen}

\begin{itemize}
    \item Aus welchen Motiven Gründen Menschen Unternehmen?Was zeichnet einen guten Entrepeneur aus?
    \begin{itemize}
        \item Unzufriedenheit mit Unternehmen
        \item Eigenständigkeit / kein Chef
        \item mehr Freiheiten als Angestellte
        \item höheres Gehalt
        \item Sinnhaftigkeit
        \item eigene Idee Umsetzten
        \item Ausnutzung Marktlücke / Nische
        \item  geringe Auslastung in einem Bereich
    \end{itemize}
    \item Was zeichnet einen guten Entrepreneur aus?
    \begin{itemize}
        \item Kritikfähig
        \item Kompetenz
        \item Work-a-holic
        \item Auslagerung von Kompetenzen
        \item hartnäckig
        \item ausdauernd
        \item Einfühlungsvermögen
        \item Begeisterungsfähigkeit
        \item Überzeugungskraft
        \item Im Interesse des Unternehmens handeln
        \item Verbesserungsfähig
        \item Fehler eingestehen
        \item kein Arschloch
        \item fair sein / beurteilen
        \item nicht nachtragend
    \end{itemize}\newpage
    \item Was sind die Vor- und Nachteile einer Unternehmensgründung?
    \begin{itemize}
        \item höheres Risiko
        \item Insolvenzrisiko
        \item Zeitintensiv
        \item hoher Verwaltungsaufwand
        \item mehr Freiheiten
        \item möglicherweise keine Kunden
        \item höheres persönliches Risiko
        \item anfänglich geringe Gewinne (wenn überhaupt) 
    \end{itemize}
    \item Schritte einer Gründung
    \begin{itemize}
        \item Kapital
        \item Idee
        \item Investment falls nötig
        \item Angestellte
        \item Prototyp
        \item Rechtsform
        \item Business Plans
        \item Kundengewinnung
    \end{itemize}
    \item Faktoren
    \begin{itemize}
        \item Schritte
        \item siehe Risiken
    \end{itemize}
    \item Gründe zum Scheitern
    \begin{itemize}
        \item Schlechte Planung
        \item Unerwartete Ereignisse
        \item Verkalkulation
        \item zu hohe Steuervorzahlungen
        \item schlechte Markteinschätzung
        \item zu hohe Kosten
        \item Mentale Probleme
        \item zu schnelles Wachstum
        \item falsche Prognosen
        \item externe Einflüsse (Katastrophen, Kriege, ...) 
    \end{itemize}
\end{itemize}
    \subsection{Unternehmensplan}
    \subsection{Die 9 Schritte zur Selbstständigkeit}
        \subsubsection{Entscheidung für die Selbstständigkeit}
            \paragraph*{Motive für die Existenzgründung}
            \begin{itemize}
                \item Innovation
                \item Anerkennung
                \item Rollenverhalten
                \item Selbstverwirklichung
                \item Unabhängigkeit
                \item Wirtschaftlicher Erfolg
            \end{itemize}

            
            \paragraph*{Eigenschaften eines Gründers}
            \begin{itemize}
                \item Flexibilität
                \item Machbarkeitsüberzeugung
                \item Risikofreudigkeit
                \item Soziale Kompetenz
                \item Entschlussfreudigkeit
                \item Problemorientierung
                \item Wachstumsorientierung
                \item Durchhaltevermögen
                \item Unabhängigkeitsstreben
                \item Leistungsmotiv
            \end{itemize}
            \paragraph*{ Auslöser der Gründungsaktivität „Theory of planned behavior“}
                „Unternehmensgründungen sind kein spontanes Ereignis zu einem zufälligen Zeitpunkt, sondern das Ergebnis von situativen und kulturellen Faktoren.“\newpage
                \begin{center}
                Äußere oder innere Lebensumstände\\+\\
                Positive Bewertung der Selbstständigkeit\\+\\
                Persönliche hohe Handlungsbereitschaft\\=\\
                Wahrscheinlichkeit der Unternehmensgründung \end{center}
        \subsubsection{Zusammenstellung eines Teams}
            \paragraph{Gründe für eine Gründung im Team}~\\
            \begin{itemize}
                \item Ausgleich der vorhandenen Schwächen (Persönlichkeit, Kompetenz, Know-how)
                \item Größere Finanzkraft
                \item Geteiltes finanzielles Risiko 
                \item Gegenseitige Sparringspartner bei Ideen- und Entscheidungsfindung
            \end{itemize}
            \paragraph{Aber...}~\\
            \begin{itemize}
                \item Verlust an Autonomie
                \item Aufteilen der Erlöse auf mehrere Gründer anteilig an Beteiligung
            \end{itemize}
            \notebox{Eine durchdachte Zusammenstellung des Gründerteams ist entscheidend!}
            \paragraph{Merkmale einer guten Teamzusammenstellung}~\\
            \begin{itemize}
                \item Teile eine gemeinsamen Vision
                \item Gemeinsame Motivation
                \item Hohe Teamfähigkeit und gegenseitige Unterstützung
                \item Offene und regelmäßige Kommunikation
                \item Komplmentäre Eigenschaften und stärken
                \item Klare Vereinbarungen über Eigenstumsverhältnisse
                \item Klare Vereinbarungen über Rechte und Pflichten
                \item Klare Aufteilung der Zuständigkeiten
            \end{itemize}

            \paragraph{Eine häufige Aufteilung in einem Start-Up ist}
            \begin{itemize}
                \item Einer übernimmt die Entwicklung/Technische Leistung
                \item der andere übernimmt den Vertrieb / die kaufmännische Leitung
            \end{itemize}
        \subsubsection{Geschäftsidee entwickeln}
        \subsubsection{Geschäftsmodell konzipieren}
        \subsubsection{Businessplan aufstellen}
        \subsubsection{Finanzierung}
        \subsubsection{Unternehmen gründen}
        \subsubsection{Angebot vermarkten}
        \subsubsection{Erfolg hinterfragen}

    
\input{Sem1/AWBL/2_AWBL_Prüfungsaufgaben}
\chapter{Medientechnik}

Der Medientechnik-bereich ist sehr groß aber als Medieninformatiker kann ich entsprechend mich nur zum Herr Hottong und Herr Reusch äußern.
\\~\\
\paragraph*{Herr Reusch}~\\ Hat viel Ahnung wovon er redet und freut sich über Teilnahme in den Vorlesungen und spezifischeren Fragen. Er arbeitet jedoch schon so lange im professionellen Audiobereich das er teilweise einige gute und günstige Angebote nicht kennt.\\ Ich hab euch mal unter \ref{Hardewareguide} Hardewareguide mal ein paar Empfehlungen zusammengepackt die recht gut sind wenn ihr euch im \glqq Budget \grqq -Audiobereich ein paar Sachen anschaffen wollt. 

\section{SPO}
    \subsection{Audiotechnik}
        \subsubsection{Lernergebnisse:}
        Nachdem Studierende das Modul erfolgreich abgeschlossen haben, können sie:

        \begin{itemize}
            \item Wissen / Kenntnisse\\
                die AV-technischen Voraussetzungen der computerbasierten Medienproduktion kennen und beherrschen
            \item Verstehen\\
                die physikalischen AV-Grundlagen in computerbasierten Medienanwendungen in Beziehung setzen
            \item Anwenden\\
                die erworbenen theoretischen und technischen Kenntnisse auf konkrete Medienanwendungen übertragen
            \item Analyse\\
                Aufgabenstellungen in computerbasierten Medienproduktionen erkennen und analysieren sowie deren Durchführung planen
            \item Synthesis\\
                infache AV-Produktionen zusammen mit computergenerierten Inhalten durchführen.
            \item Evaluation\\
            etwaige Fehler im computerbasierten AV-Produktionsprozess erkennen und korrigieren.\\
            sicher mit AV-Produktionsequipment umgehen
        \end{itemize}
\newpage
\section{Audiotechnik Reusch}

\input{Sem1/Medientechnik/Audiotechnik_Reusch/Einstieg_Reusch}
\input{Sem1/Medientechnik/Audiotechnik_Reusch/Schallwellen_Reusch}
\input{Sem1/Medientechnik/Audiotechnik_Reusch/Ohr_Reusch}
\input{Sem1/Medientechnik/Audiotechnik_Reusch/Mikrofone_Reusch}
\input{Sem1/Medientechnik/Audiotechnik_Reusch/Lokalisation_Reusch}
\input{Sem1/Medientechnik/Audiotechnik_Reusch/Stereo_reusch}
\input{Sem1/Medientechnik/Audiotechnik_Reusch/Mikrofonierung_Reusch}
\input{Sem1/Medientechnik/Audiotechnik_Reusch/Anschlusstechnik_Reusch}
\input{Sem1/Medientechnik/Audiotechnik_Reusch/DAW_Reusch}
\input{Sem1/Medientechnik/Audiotechnik_Reusch/Aufzeichnung_Reusch}
\input{Sem1/Medientechnik/Audiotechnik_Reusch/Praxisbeispiele}
\input{Sem1/Medientechnik/Audiotechnik_Reusch/Audiotechnik_Beispiele}






\newpage
\section{Videotechnik Hottong}

\input{Sem1/Medientechnik/Videotechnik_Hottong/Licht_Hottong}
\input{Sem1/Medientechnik/Videotechnik_Hottong/Objektive_Hottong}
\input{Sem1/Medientechnik/Videotechnik_Hottong/Wahrnehmung_hottong}
\input{Sem1/Medientechnik/Videotechnik_Hottong/Techn_Qualitaet_Hottong}
\input{Sem1/Medientechnik/Videotechnik_Hottong/Farbmodelle_Hottong}
\input{Sem1/Medientechnik/Videotechnik_Hottong/Generierung_Bilder_Hottong}
\input{Sem1/Medientechnik/Videotechnik_Hottong/Bildaufnahmetechniken_Hottong}

\chapter{Gestaltung}
    \section{SPO}
    Nachdem Studierende das Modul erfolgreich abgeschlossen haben, können sie:
    \begin{itemize}
        \item Wissen / Kenntnisse\\
            Die Grundlagen gestalterischer Fragestellungen beurteilen.
            \newline
            Theorien zur Medienrezeption benennen.
        \item Verstehen\\
            Kreative Prozesse verstehen und selber erste Gestaltarbeiten anfertigen.
            \newline
            Verstehen, wo wir als Rezipienten und als Produzenten auf wissenschaftliche Erkenntnisse aufbauen können.
        \item Anwenden\\
            Erste Konzeptionen entwickeln und mit den Augen eines Gestalters Kreativarbeit beurteilen. \newline
            Medienpsychologische Theorien anwenden.
        \item Analyse\\
            Gestaltungsparameter untersuchen und Produktionsprozesse darstellen.
            \newline
            Medienpsychologische Prozesse analysieren.
    \end{itemize}
    
\newpage
    \section{Mediengestaltung}

\newpage
    \section{Medienpsychologie}

\newpage
\newpage
\chapter{Programmieren}

Dieser Kurs ist \textbf{NICHT} für komplette Programmieranfänger ohne großen Zeitaufwand zu bewältigen. Der Eisenbiegler ist ein geborener Informatiker, jedoch leider nicht ein guter Professor. Ich empfehle euch Java unabhängig von seinen Resourcen zu lernen, seine Übungsaufgaben sind jedoch Sinnvoll konzipiert. Es steigen regelmäßig über mehrere Semester mehrere Personen, meist ohne Vorkentnisse, an der Bildverarbeitungsaufgabe aus.\\
Wenn ihr jedoch schon Vorkentnisse habt kann euch der Herr Eisenbiegler gut weiterhelfen um entsprechend eure Kenntnisse zu vertiefen und euch Sinnvolle Strukturen an die Hand geben.\\
Ich persönlich bin halt auch zwischendurch Ausgestiegen und er konnte mir persönlich nicht die Hilfe bieten die ich gebraucht habe. :/ \\
% nicht sehr motivierend bro
Wer einen Onlinekurs nebenbei belegen will empfehle ich Kurse von \href{https://open.hpi.de/courses}{openhpi}, dies sind kostenfreie Kurse. Speziell empfehle ich diesen \href{https://open.hpi.de/courses/javaeinstieg2020}{Java Kurs}, als alternative zu diesem gibt es auch noch die  \href{https://open.hpi.de/courses/javaeinstieg-schule2024}{Schülerversion}. Dieser wurde mir jedoch von einer Professorin aus der Bekanntschaft empfohlen. \\ Ein weiterer Anbieter von solchen Kursen ist \href{https://www.codecademy.com/learn/learn-java}{CodeCademy}, diese beiten unter anderem auch kostenlose urse zu \href{https://www.codecademy.com/learn/learn-java}{Java} oder weiteren wie \href{https://www.codecademy.com/learn/learn-python-3}{Python 3}.
\\~\\
\emph{- Nils}

\newpage
\section{SPO}

    Nachdem Studierende das Modul erfolgreich abgeschlossen haben, können sie:
    \begin{itemize}
        \item Wissen / Kenntnisse\\
            Die Sprachelemente einer imperativen Programmiersprache benennen.
        \item Verstehen\\
            Die Bedeutung eines imperativen Computerprogramms erklären.
        \item Anwenden\\
            Mit einer integrierten Entwicklungsumgebung arbeiten.
        \item Analyse\\
            Den Ablauf eines vorgegebenen imperativen Computerprogramms beschreiben.
        \item Synthesis\\
            Zu einer einfachen Aufgabenstellung ein imperatives Computerprogramm selbstständig implementieren.
        \item Evaluation\\
            Unterschiedliche Computerprogramme in Bezug auf ihre Effizienz miteinander vergleichen.
    \end{itemize}
    
\newpage
\subsection{Theorie}\newpage
\subsection{Übungsaufgaben}

\input{Sem1/Programmieren/Uebungen_Eisenbiegler/A1P_Einführung}
\input{Sem1/Programmieren/Uebungen_Eisenbiegler/A2P_ImperativeProg}
\input{Sem1/Programmieren/Uebungen_Eisenbiegler/A3P_ImperativeProg}
\input{Sem1/Programmieren/Uebungen_Eisenbiegler/A4P_Bildverarbeitung}
\input{Sem1/Programmieren/Uebungen_Eisenbiegler/A5P_ImperativeProg}
\input{Sem1/Programmieren/Uebungen_Eisenbiegler/A6P_Rekursion}
\input{Sem1/Programmieren/Uebungen_Eisenbiegler/A7P_Objekte}
\input{Sem1/Programmieren/Uebungen_Eisenbiegler/A8P_VerketteteObjekte}
\input{Sem1/Programmieren/Uebungen_Eisenbiegler/A9P_ExceptionHandle}
\input{Sem1/Programmieren/Uebungen_Eisenbiegler/A10P_ExceptionsAbst}
\input{Sem1/Programmieren/Uebungen_Eisenbiegler/A11P_PersonenImKrankenhaus}
\input{Sem1/Programmieren/Uebungen_Eisenbiegler/A12P_Fensterbestellung}

\newpage
\section{CodeCademy Notizen}

Hi, Nils Hier. Ich bin mirt nicht sicher wie ich den Inhalt vom Herrn Eisenbiegler gut weitergeben kann, vorallem da ich hier auch gestruggelt habe. Darum habe ich hier Angefangen meinen Vortschritt des CodeCademy Java Kursen in Notizen zu fassen um es Inhaltlich sinnvoll wiederzugeben. Zusätzlich habe ich den Abscnitt hier noch einmal von Daniel rüber schauen lassen, da er zumindest mehr Verständis von Code hat als ich.

\subsection{Lesson 1 - Hello Word.}
\paragraph{Java's entstehung}~\\
Zu aller erst einmal etwas zu der Programmiersprache, Java wurde 1995 von Sun Microsystems veröffentlicht. Java ist dafür bekannt einfach, sicher, robust und portierbar zu sein. Dies macht es heute immer noch zu einer der beliebtesten Programmiersprachen bis heute.

\paragraph{Java}~\\
Was Java so gut macht ist unter anderem ihre VM (Virtual Machiene), die es erlaubt Java auf eigentlich jeder Plattform zu laufen.\\
Programmiersprachen sind mit Hilfe der \textit{Syntaxe} aufgebaut. Wir schreiben diese in Datein, welche dann von dem PC die einprogrammierte Aufgabe ausführt. 

\paragraph{Hello World - jetzt aber Wirklich}~\\
\begin{verbatim}
public class HelloWorld {
  public static void main(String[] args) {
    System.out.println("Hello World!");
  }
}
\end{verbatim}

Was hier zu beachten ist das die \textit{class} sozusagen das Gebiet bezeichnet, (wie eine Adresse) und alles in dieser Adresse ne Fabrik ist die eine oder viele Aufgaben erfüllt.\\
Wichtig hier ist jedoch das wenn die Datei \glqq Hello World \grqq heißt, muss entsprechend auch die Klasse \glqq Hello World \grqq heißen. Der Name der Klasse ist immer der gleiche wie der der Datei. 
\subsection{Lesson 2 - Java Datein}
Java Datein enden immer mit der endung \glqq .java \grqq
In dieser Datei steht eine Klasse, im letzten Beispiel: 

\begin{verbatim}
public class HelloWorld {
  
}
\end{verbatim}
Santaxe in den geschweiften Klammern \glqq \{ \}\grqq~gehören zu der zugehörigen Klasse.\\
Jede Datei hat eine Klasse die nach der Datei benannt ist, dies ist dann die Hauptklasse. Die Bennenungskonvention ist in der regel so, das jedes neue Wort groß geschrieben wird um es verständlich zu machen. So heißt dann die Datei: \glqq HelloWold.java\grqq~und die Klasse: \glqq HelloWolrd\grqq.\\~\\
In unserer Klasse hatten wir eine main()-Methode, welche unsere Anwendungsaufganen aufliestet:
\begin{verbatim}
public static void main(String[] args) {

}
\end{verbatim}

Dieses Programm hatte die Aufgabe \glqq Hello World \glqq~ auszugeben, was durch den folgenden Befehl erreicht wurde: 
\begin{verbatim}
    System.out.println("Hello World");
\end{verbatim}
\subsection{Lesson 3 - Print statements}
Hier schauen wir etwas tiefer in die Funktionsweise des Print-Statements rein.
\begin{verbatim}
    System.out.println("Hello World");
\end{verbatim}
\begin{itemize}
    \item System\\
    ist in Java eingebaut und bietet nützliche Tools für unser Programm
    \item out\\
    ist die abkürzung von Output
    \item println\\
    steht für "printline"
\end{itemize}
Wir können \glqq System.out.println() \grqq~immer verwenden um
\subsection{Lesson x - x}

\subsection{Lesson x - x}

\subsection{Lesson x - x}

\subsection{Lesson x - x}

\subsection{Lesson x - x}

\subsection{Lesson x - x}

\subsection{Lesson x - x}

\subsection{Lesson x - x}

\subsection{Lesson x - x}

\subsection{Lesson x - x}

\subsection{Lesson x - x}

\subsection{Lesson x - x}

\subsection{Lesson x - x}

\subsection{Lesson x - x}

\subsection{Lesson x - x}



Java Glossar:

\begin{itemize}
    \item Class
    \item 
    \item 
    \item 
    \item 
    \item 
    \item 
    \item 
    \item 
    \item 
    \item 
    \item 
    \item 
    \item 
    \item 
    
\end{itemize}

\chapter{MINT}
    \section{SPO}

    Nachdem Studierende das Modul erfolgreich abgeschlossen haben, können sie:
    \begin{itemize}
        \item Wissen / Kenntnisse\\
            Geometrische und algebraische Fragestellungen präzise mithilfe der adäquaten Fachbegriffe artikulieren. 
            \newline
            Zentrale Grundbegriffe der Optik sicher wiedergeben.
        \item Verstehen\\
            Mathematische Sinnzusammenhänge und Beweiselemente bzw. Herleitungen erkennen verstehen und wiedergeben. 
            \newline
            Mathematische Modelle physikalischer Phänomene (z.B. geometrisch-optisches paraxiales Arbeitsmodell für abbildende Systeme) verstehen.
        \item Anwenden\\
            Techniken der Vektorrechnung und der Matrixalgebra auf geometrische Probleme anwenden. \newline
            Grundgesetze der Strahlenoptik auf einfache Kameraobjektivmodelle bzw. Fragestellungen der Fotografie anwenden.
        \item Analyse\\
            Geometrische Standardprobleme in der Ebene und im Raum analysieren. 
            \newline
            Angemessen ausgewählte physikalische Systeme und Strukturen selbstständig analysieren und beschreiben.
        \newpage
        \item Synthesis\\
            Für Frage- und Problemstellungen aus (Linearer) Algebra und Geometrie unter den bereitgestellten Hilfsmitteln die jeweils adäquaten auswählen. 
            \newline
            Ein geeignetes eingegrenztes, für die Medientechnik relevantes Thema aus Optik oder Akustik im Überblick darstellen.
        \item Evaluation\\
            Verschiedene Verfahren (z.B. zur Bestimmung affiner Transformationen) hinsichtlich Übersichtlichkeit und Aufwand abwägen.
    \end{itemize}
    
\newpage
    \section{Mathe}
    \begin{itemize}
        \item {Grundlegende Kenntnisse}
        \item{Einheitskreis}
    \end{itemize}
    
    \subsection{Einleitung}
    \input{Sem1/MINT/Mathe_Lasowski/Grundlegendes}
    \input{Sem1/MINT/Mathe_Lasowski/Einstieg_und_Wiederholung}
\newpage
    \section{Physik}
    \begin{itemize}
        \item{Einleitung}
        \item{Optik abbildender Systeme}
    \end{itemize}

    \input{Sem1/MINT/Physik_Garcia/Einleitung}
    \input{Sem1/MINT/Physik_Garcia/Optik_abbildender_Systeme}
    Keine Ahnung :(
\newpage
    \input{Sem1/MINT/2_MINT_Prüfungsaufgaben}
\newpage
\chapter{weitere Inhalte}

\section{Guide} 
Im folgenden sind Produkte vorgestellt die ich empfehlen kann, jedoch unterscheidet sich der genaue Fall von Person zu Person.

\subsection{Hardware}   \label{Hardewareguide}
Noch kurz davor, es gibt viele Methoden um Produkte entsprechend anzuschließen und es ist von Person zu Person unterschiedlich. Dementsprechend sind meine allgemeine Empfehlung \textbf{Fett} markiert, diese Produkte passen entsprechen dann auch zusammen. Ich füge jedoch immer eine kleine Beschreibung hinzu um meine Gedanken etwas zu erklären. Hier geht es jedoch um den Heimstudiobereich und entsprechend sind die Produkte auch so ausgesucht. 

\subsubsection{Interface}
Ein Interface ist die Schnittstelle zwischen PC und euren Geräten, es gibt viele Varianten aber ein Geräöt mit USB macht sich dabei einfach praktisch und kann auch sehr gute Qualität haben.\\

\begin{itemize}
    \item \textbf{\href{https://www.thomann.de/de/focusrite_scarlett_solo_3rd_gen.htm}{Focusrite Scarlett Solo}}\\
    ist ein gutes USB-Interface was eins der Standart Produkte ist. Es gibt auch Varianten für mehr Geld welche entsprechnd mehr Anschlüsse haben.
    \item \href{https://www.thomann.de/de/behringer_u_phoria_umc22.htm}{Behringer U-Phoria UMC22}\\ ist so eins der günstigen alternativen zum Focusrite.
\end{itemize}

\subsubsection{Mikrofone}
Um ein gutes Mikrofon kommt man bei keiner Aufnahme herum. Es gibt hier diesmal 2 Empfehlungen, wer sich nicht sicher ist informiert sich bitte noch einmal unter den Mikrofonen um die Unterschiede. \\

\begin{itemize}
    \item \textbf{\href{https://www.thomann.de/de/lewitt_lct_240_pro_bk_bundle.htm}{Lewitt LCT 240 Pro}}\\    ist wohl eins der Preis-/Leistung besten Kondensatormikrofone die es so auf dem Markt gibt. Ich würde direkt dieses Bundle mit Spinne kaufen, der Unterschied lohnt sich aufgrund der hohen Sensivität und es ist extrem Robust, es läuft bei mir seit über 5 Jahren nun schon ohne Probleme. Hier wird jedoch 48V Phantomspannung benötigt. 
    \item \textbf{\href{https://www.thomann.de/de/shure_sm58s.htm}{Shure SM58 S}}\\ schlichtweg der Standart bei Dynamischen Mikrofonen, kann ich nur empfehlen. Das \href{https://www.thomann.de/de/shure_sm58s.htm}{SM58 S} empfehle ich über das normale \href{https://www.thomann.de/de/shure_sm58.htm}{SM58} aufgrund des Schalters um bei Bedarf sich selbst noch Stumm zu schalten. 
    \item \href{https://www.thomann.de/de/behringer_ba_85a.htm}{Behringer BA 85A} \\ einfach eine günstige Alternative.

\end{itemize}


\subsubsection{Kopfhörer}
Es gibt offene und geschlossene Kopfhörer, offene sind "klarer" aber lassen Umgebungsgeräusche zu. Geschlossene sind der Standart im Studio. Hier sind alle Empfehlungen, weil es lohnt sich hier nicht günstiger zu gehen. Nimm sosnt deine Vorhandenen. \\Falls ihr Geld Sparen wollt schaut mal bitte hier rei: \href{https://www.beyerdynamic.de/outlet.html?gad_source=1&gclid=Cj0KCQjwiuC2BhDSARIsALOVfBJ6sfr7q1aL37ZvTAGywnzx0hW6F8qXNQv-qk8qNju73Reqe1Cvn0AaAgraEALw_wcB}{Beyerdynamic B-Stock}, die haben gute Produkte (oft ohne schaden) teilweise deutlich günstiger. Kauft wenn verfügbar bitte hier.
\begin{itemize}
    \item \href{https://www.thomann.de/de/beyerdynamic_dt770_pro80_ohm.htm}{Beyerdynamic DT-770 Pro 80 Ohm}\\ eigentlich der Preis Leistungsstandart, immer ne Empfehlung. Beyerdynamic bietet auch eigentlich für alles ersatzteile an. Dies ist die 80 Ohm Variante, ist noch ohne Verstärker/Interface laufbar, ansonsten gerne die hier kaufen \href{https://www.thomann.de/de/beyerdynamic_dt770pro.htm}{DT-770 Pro 250 Ohm}.
    \item \href{https://www.thomann.de/de/sony_mdr7506_kopfhoerer.htm}{Sony MDR-7506}\\ das ist so das andere Equivalent von Sony zu den DT-770. 
    \item \href{https://www.thomann.de/de/beyerdynamic_dt990pro.htm}{Beyerdynamic DT-990 Pro} \\ Dies hier ist die offene Variante, bietet klareren klang aber auch Umgebungsgeräusche werden gehört. Hier bitte nur die 250 Ohm Variante holen, die andere lohnt sich nicht. 
\end{itemize}


\section{Softwareguide}
Hier findet ihr Software oder Add-Ons die ich persönlich benutze und weiterempfehlen kann. Nicht alle davon werden auf allen Betriebssystemen laufen oder geeignet sein, ich persönlich benutze eine Windows-Maschine im regulären Betrieb. Entsprechend sind diese sehr Biased, ich habe versucht alternativen für andere Plattformen durch Freunde und eigener Recherche zu finden, jedoch bin ich halt da nicht so in der Materie drin.\\
Ich versuche auch möglichst immer Kostenlose Software auszuwählen da da diese für Studenten recht geil sind, wo dies nicht der Fall ist steht das dabei.

\subsubsection{Softwareauflistung}
\begin{itemize}
    \item \href{https://www.reaper.fm}{Reaper}\\
    Reaper ist die Digital Audio Workstation die Herr Reusch persönlich nutzt und empfiehlt, diese ist kostenlos nutzbar (auch nach den 60 Tagen Probezeitraum). Bitte kauft eine Lizenz (60€) wenn ihr beginnt Geld damit zu verdienen. Hier sind auch eigentlich alle VST-Plugins verwendbar und es sind für fast alle Effekte schon gute Vorinstalliert.
    \item \href{https://apps.apple.com/de/app/logic-pro/id634148309}{Apple Logic Pro}\\
    Die Wohl beste DAW, sie kommt mit allen aus der Box. Wer ein Apple Gerät hat und die 230€ dafür hat, ist diese DAW wohl die beste die es gibt. Alle Effekte sind gut und es ist auch eine sehr gute Bibliothek mit Instumenten vorhanden.
    \item \href{https://vb-audio.com/Voicemeeter/index.htm}{VB-Audio und Software}\\
    Voicemeeter ist ein digitales Mischpult und Audio-Routing Lösung, die Standort Variante "Voicemeeter" ist kostenlos und bietet bessere Einstellungen als Standard Windows. Es bietet mehrere Analoge Ein-und Ausgänge sowie digitale Äquivalente als Standard Windows. Darüber hinaus ist die "Potato"-Variante ab 10\$ erhältlich und bietet dir den größten Umfang. Dazu kannst du auch via Netzwerk (LAN und WLAN) Audio im lokalen Netz übertragen. VB-Audio bietet dazu auch noch eine weitere Bibliothek von Zusatzprodukten um diese herum an.
    \item \href{https://apps.ankiweb.net}{Anki}\\
    Anki ist eine Kostenlose Dateikartensoftware welche gut zum lernen benutzt werden kann.
    \item \href{https://obsproject.com/de/download}{OBS-Studio}\\
    Ist eigentlich die Standard-Software was Aufnahmen  und Livestreaming angeht, wenn man dies vorhat eigentlich ein muss.
    \item \href{https://www.videolan.org/vlc/index.de.html}{VLC Media Player}\\
    Eigentlich der Standart-Media Player da dieser nahezu alle Datein öffnen kann und auch Medienkonventierung ist recht geil.
    \item \href{https://rustdesk.com}{RustDesk}
    Eine Alternative zu TeamViewer als RemoteDesktop App, in Rust geschrieben und bietet aus meiner Ansicht nach bessere Performance.
    \item \href{https://www.blackmagicdesign.com/de/products/davinciresolve}{DaVinci Resolve}\\
    Ist eine gute Video- und VFX-Software, welche nahezu ohne Einschränkungen benutzbar ist. Die Einmal-Lizenz ist durch den Kauf (250€) oder durch den Erwerb eines gekennzeichneten \href{https://www.blackmagicdesign.com/de/products}{Blackmagic} Produktes wie eine Kamera erhältlich. Für den Download ist das eintragen persönlicher Daten notwendig, also nicht wundern.
    \item \href{https://obsidian.md}{Obsidian.md}\\
    Ist eine Notizapp, welche auf allen Plattformen verfügbar ist. Notizen werden im Markdown-Format erfasst und das Ziel ist der Aufbau eines zettelkastens. Dadurch sollen Themen besser durch Querverweise miteinander verknüpft werden. Dies ist mein persönliche 
    \item \href{https://www.pureref.com/download.php}{PureRef}\\
    Ist ein kleines pop-Up Window welches in dem Vordergrund geheftet werden kann, ist ganz geil für Laptops um einen weiteren permanenten Bildschirmbereich für nahezu alle Bildvorlage zu öffnen.
    \item \href{https://getsharex.com/downloads}{ShareX}\\
    Meiner Meinung nach das beste Screenshot-Tool auf dem Markt, er kommt auch mit einem recht guten Editor um leichte Anpassungen wie Zuschnitte oder Highlights hinzuzufügen.
    \item \href{https://bitwarden.com/de-de/download/#downloads-desktop}{Bitwarden} \& \href{https://keepassxc.org/download/#windows}{KeypassXC}\\
    Dies sind 2 gute Passwortmanager, eigentlich ein muss wenn man noch keine Verwendet. Bitwarden hat den Vorteil der Online-Synchronisation und kann über einem Selfhostet Vaultwarden selbst gesynct werden. In KeypassXC muss die Passwortdatei stehts ausgetauscht werden aber sehr beliebt und hat eine Browserextention. Wer keinen Passwortmanager hat sollte sich die überlegen zu benutzten.
    \item \href{https://freeotp.github.io/}{FreeOTP} \& Alternativen\\
    Eine "-factor Authentifizierungssoftware, ein generell guter Schritt in Richtung Sicherheit seiner Dienste. Grundsätzlich sollten auch SMS-Verifizierungen vermieden werden und dies ist eine sehr gute Alternative.
    \item \href{https://de.libreoffice.org/download/download/}{LibreOffice}, \href{https://www.openoffice.org/de/download/}{OpenOffice} \& \href{https://proton.me/de/drive/docs}{ProtonDocs}\\
    Libre- und OpenOffice sind eine alternative zu Microsoft Office, jedoch ist diese für Studenten kostenlos verfügbar. ProtonDocs ist eine Alternative zu Google Workspace, jedoch gibt es bis jetzt nur Textdokumentsupport. 
    \item \href{https://code.visualstudio.com/download}{VS-Code}, \href{https://www.eclipse.org/downloads/}{Eclipse} \& \href{https://www.jetbrains.com/idea/}{IntelliJ IDEA}\\ Alle Editoren sind für Studenten frei zugänglich, IntelliJ's jedoch erst nach Aktivierung. Dies sind so mit die Empfohlenen Code-Editoren, es hängt von euch ab welchen ihr gerne benutzt.
    \item \href{https://education.github.com/discount_requests/application}{GitHub} \& \href{https://desktop.github.com/download/}{GitHub Desktop}\\
    GitHub ist eine der gängigen Versionierungs- und Code-Sharing-Platform, sie haben ein Studentenprogramm welchem einen Zugrif für viele weitere Softwares gibt. Dort ist unter anderem auch der KI-Codeeditor Copilot verfügbar. Die Desktop App bietet mit die beste und einsteigefreundlichste Variante um seine Datein mit GitHub zu synchronisieren und hat ein gutes Tutorial dazu.
    \item \href{https://www.blender.org/}{Blender} \& \href{https://manage.autodesk.com/products/maya}{Maya}\\
    Blender ist eine der gängigsten 3D-Software für die Erstellung von Modells, jedoch unterscheidet sie sich zum teil drastisch von den Industriestandarts. Maya von Autodesk st einder dieser, um diesen zu benutzten muss man sich als Student verifizieren.
    \item \href{https://krita.org/de/download/}{Krita}, \href{https://krita.org/de/download/}{Gimp}, \href{https://www.clipstudio.net/de/dl/}{Clip Studio Paint}\\
    Diese 3 Softwares sind ein Stück weit alternativen zu Photoshop im Bereich der Zeichenprogramme, die ersten beiden sind frei und Clip Studio kostet einmalig Geld um es zu benutzten.
    \item \href{https://inkscape.org/release/inkscape-1.3.2/windows/64-bit/msi/?redirected=1}{Inkscape}\\ Ist eine gute Alternative für Adobe Illustrator, er bietet viele möglichkeiten Vektorgrafiken zu erstellen und zu bearbeiten.
    \item \href{https://discord.com/download}{Discord}\\
    Discord ist eine Text- und Videochatsoftware, dies ist der standart bei Gamern und sonst eigentlich auch in der Hochschule Recht häufig verwendet. 
    \item \href{https://www.adobe.com/de/creativecloud/buy/students.html?promoid=65FN7X8B&mv=other}{Adobe}, \href{https://affinity.serif.com/de/}{Affinity} \& \href{https://www.magix.com/de/education/}{Magix}\\
    Diese 3 unternehmen bieten Softwarebundels im Kreativbereich an, welche auch viel genutzt werden. Die meisten haben eine Studentenoption, auch sind Magix und Affinity duch eine Einmalzahlung erwerbbar, wodurch sie auf lange dauer deutlich günstiger als Adobe sind. Hier sollte man sich jedoch selbst reinarbeiten, weil diese sind für das Studium \textbf{nicht} nötig.
\end{itemize}

\input{weitere_Inhalte/Bücherliste}


% ======= Quellen %     hinzufügen vom Quellendokument

% Diese bitte nicht Auskommendtieren, es werden nur die Angezeigt die im Dokument vorkommen.

\addcontentsline{toc}{chapter}{Quellen}
\addcontentsline{toc}{section}{Weblinks}
\printbibliography[type=online,title={Weblinks}]% nur online-artikel
\printbibliography[type=book, title={Bücher}]   % nur Bücher 
\printbibliography[type=misc, title={andere}]

\addcontentsline{toc}{section}{Artikel}
\printbibliography[type=article,title={Articles only}]

\end{document}