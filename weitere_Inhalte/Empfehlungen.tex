\section{Guide} 
Im folgenden sind Produkte vorgestellt die ich empfehlen kann, jedoch unterscheidet sich der genaue Fall von Person zu Person.

\subsection{Hardware}   \label{Hardewareguide}
Noch kurz davor, es gibt viele Methoden um Produkte entsprechend anzuschließen und es ist von Person zu Person unterschiedlich. Dementsprechend sind meine allgemeine Empfehlung \textbf{Fett} markiert, diese Produkte passen entsprechen dann auch zusammen. Ich füge jedoch immer eine kleine Beschreibung hinzu um meine Gedanken etwas zu erklären. Hier geht es jedoch um den Heimstudiobereich und entsprechend sind die Produkte auch so ausgesucht. 

\subsubsection{Interface}
Ein Interface ist die Schnittstelle zwischen PC und euren Geräten, es gibt viele Varianten aber ein Geräöt mit USB macht sich dabei einfach praktisch und kann auch sehr gute Qualität haben.\\

\begin{itemize}
    \item \textbf{\href{https://www.thomann.de/de/focusrite_scarlett_solo_3rd_gen.htm}{Focusrite Scarlett Solo}}\\
    ist ein gutes USB-Interface was eins der Standart Produkte ist. Es gibt auch Varianten für mehr Geld welche entsprechnd mehr Anschlüsse haben.
    \item \href{https://www.thomann.de/de/behringer_u_phoria_umc22.htm}{Behringer U-Phoria UMC22}\\ ist so eins der günstigen alternativen zum Focusrite.
\end{itemize}

\subsubsection{Mikrofone}
Um ein gutes Mikrofon kommt man bei keiner Aufnahme herum. Es gibt hier diesmal 2 Empfehlungen, wer sich nicht sicher ist informiert sich bitte noch einmal unter den Mikrofonen um die Unterschiede. \\

\begin{itemize}
    \item \textbf{\href{https://www.thomann.de/de/lewitt_lct_240_pro_bk_bundle.htm}{Lewitt LCT 240 Pro}}\\    ist wohl eins der Preis-/Leistung besten Kondensatormikrofone die es so auf dem Markt gibt. Ich würde direkt dieses Bundle mit Spinne kaufen, der Unterschied lohnt sich aufgrund der hohen Sensivität und es ist extrem Robust, es läuft bei mir seit über 5 Jahren nun schon ohne Probleme. Hier wird jedoch 48V Phantomspannung benötigt. 
    \item \textbf{\href{https://www.thomann.de/de/shure_sm58s.htm}{Shure SM58 S}}\\ schlichtweg der Standart bei Dynamischen Mikrofonen, kann ich nur empfehlen. Das \href{https://www.thomann.de/de/shure_sm58s.htm}{SM58 S} empfehle ich über das normale \href{https://www.thomann.de/de/shure_sm58.htm}{SM58} aufgrund des Schalters um bei Bedarf sich selbst noch Stumm zu schalten. 
    \item \href{https://www.thomann.de/de/behringer_ba_85a.htm}{Behringer BA 85A} \\ einfach eine günstige Alternative.

\end{itemize}


\subsubsection{Kopfhörer}
Es gibt offene und geschlossene Kopfhörer, offene sind "klarer" aber lassen Umgebungsgeräusche zu. Geschlossene sind der Standart im Studio. Hier sind alle Empfehlungen, weil es lohnt sich hier nicht günstiger zu gehen. Nimm sosnt deine Vorhandenen. \\Falls ihr Geld Sparen wollt schaut mal bitte hier rei: \href{https://www.beyerdynamic.de/outlet.html?gad_source=1&gclid=Cj0KCQjwiuC2BhDSARIsALOVfBJ6sfr7q1aL37ZvTAGywnzx0hW6F8qXNQv-qk8qNju73Reqe1Cvn0AaAgraEALw_wcB}{Beyerdynamic B-Stock}, die haben gute Produkte (oft ohne schaden) teilweise deutlich günstiger. Kauft wenn verfügbar bitte hier.
\begin{itemize}
    \item \href{https://www.thomann.de/de/beyerdynamic_dt770_pro80_ohm.htm}{Beyerdynamic DT-770 Pro 80 Ohm}\\ eigentlich der Preis Leistungsstandart, immer ne Empfehlung. Beyerdynamic bietet auch eigentlich für alles ersatzteile an. Dies ist die 80 Ohm Variante, ist noch ohne Verstärker/Interface laufbar, ansonsten gerne die hier kaufen \href{https://www.thomann.de/de/beyerdynamic_dt770pro.htm}{DT-770 Pro 250 Ohm}.
    \item \href{https://www.thomann.de/de/sony_mdr7506_kopfhoerer.htm}{Sony MDR-7506}\\ das ist so das andere Equivalent von Sony zu den DT-770. 
    \item \href{https://www.thomann.de/de/beyerdynamic_dt990pro.htm}{Beyerdynamic DT-990 Pro} \\ Dies hier ist die offene Variante, bietet klareren klang aber auch Umgebungsgeräusche werden gehört. Hier bitte nur die 250 Ohm Variante holen, die andere lohnt sich nicht. 
\end{itemize}

