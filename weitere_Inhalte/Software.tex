\section{Softwareguide}
Hier findet ihr Software oder Add-Ons die ich persönlich benutze und weiterempfehlen kann. Nicht alle davon werden auf allen Betriebssystemen laufen oder geeignet sein, ich persönlich benutze eine Windows-Maschine im regulären Betrieb. Entsprechend sind diese sehr Biased, ich habe versucht alternativen für andere Plattformen durch Freunde und eigener Recherche zu finden, jedoch bin ich halt da nicht so in der Materie drin.\\
Ich versuche auch möglichst immer Kostenlose Software auszuwählen da da diese für Studenten recht geil sind, wo dies nicht der Fall ist steht das dabei.

\subsubsection{Softwareauflistung}
\begin{itemize}
    \item \href{https://www.reaper.fm}{Reaper}\\
    Reaper ist die Digital Audio Workstation die Herr Reusch persönlich nutzt und empfiehlt, diese ist kostenlos nutzbar (auch nach den 60 Tagen Probezeitraum). Bitte kauft eine Lizenz (60€) wenn ihr beginnt Geld damit zu verdienen. Hier sind auch eigentlich alle VST-Plugins verwendbar und es sind für fast alle Effekte schon gute Vorinstalliert.
    \item \href{https://apps.apple.com/de/app/logic-pro/id634148309}{Apple Logic Pro}\\
    Die Wohl beste DAW, sie kommt mit allen aus der Box. Wer ein Apple Gerät hat und die 230€ dafür hat, ist diese DAW wohl die beste die es gibt. Alle Effekte sind gut und es ist auch eine sehr gute Bibliothek mit Instumenten vorhanden.
    \item \href{https://vb-audio.com/Voicemeeter/index.htm}{VB-Audio und Software}\\
    Voicemeeter ist ein digitales Mischpult und Audio-Routing Lösung, die Standort Variante "Voicemeeter" ist kostenlos und bietet bessere Einstellungen als Standard Windows. Es bietet mehrere Analoge Ein-und Ausgänge sowie digitale Äquivalente als Standard Windows. Darüber hinaus ist die "Potato"-Variante ab 10\$ erhältlich und bietet dir den größten Umfang. Dazu kannst du auch via Netzwerk (LAN und WLAN) Audio im lokalen Netz übertragen. VB-Audio bietet dazu auch noch eine weitere Bibliothek von Zusatzprodukten um diese herum an.
    \item \href{https://apps.ankiweb.net}{Anki}\\
    Anki ist eine Kostenlose Dateikartensoftware welche gut zum lernen benutzt werden kann.
    \item \href{https://obsproject.com/de/download}{OBS-Studio}\\
    Ist eigentlich die Standard-Software was Aufnahmen  und Livestreaming angeht, wenn man dies vorhat eigentlich ein muss.
    \item \href{https://www.videolan.org/vlc/index.de.html}{VLC Media Player}\\
    Eigentlich der Standart-Media Player da dieser nahezu alle Datein öffnen kann und auch Medienkonventierung ist recht geil.
    \item \href{https://rustdesk.com}{RustDesk}
    Eine Alternative zu TeamViewer als RemoteDesktop App, in Rust geschrieben und bietet aus meiner Ansicht nach bessere Performance.
    \item \href{https://www.blackmagicdesign.com/de/products/davinciresolve}{DaVinci Resolve}\\
    Ist eine gute Video- und VFX-Software, welche nahezu ohne Einschränkungen benutzbar ist. Die Einmal-Lizenz ist durch den Kauf (250€) oder durch den Erwerb eines gekennzeichneten \href{https://www.blackmagicdesign.com/de/products}{Blackmagic} Produktes wie eine Kamera erhältlich. Für den Download ist das eintragen persönlicher Daten notwendig, also nicht wundern.
    \item \href{https://obsidian.md}{Obsidian.md}\\
    Ist eine Notizapp, welche auf allen Plattformen verfügbar ist. Notizen werden im Markdown-Format erfasst und das Ziel ist der Aufbau eines zettelkastens. Dadurch sollen Themen besser durch Querverweise miteinander verknüpft werden. Dies ist mein persönliche 
    \item \href{https://www.pureref.com/download.php}{PureRef}\\
    Ist ein kleines pop-Up Window welches in dem Vordergrund geheftet werden kann, ist ganz geil für Laptops um einen weiteren permanenten Bildschirmbereich für nahezu alle Bildvorlage zu öffnen.
    \item \href{https://getsharex.com/downloads}{ShareX}\\
    Meiner Meinung nach das beste Screenshot-Tool auf dem Markt, er kommt auch mit einem recht guten Editor um leichte Anpassungen wie Zuschnitte oder Highlights hinzuzufügen.
    \item \href{https://bitwarden.com/de-de/download/#downloads-desktop}{Bitwarden} \& \href{https://keepassxc.org/download/#windows}{KeypassXC}\\
    Dies sind 2 gute Passwortmanager, eigentlich ein muss wenn man noch keine Verwendet. Bitwarden hat den Vorteil der Online-Synchronisation und kann über einem Selfhostet Vaultwarden selbst gesynct werden. In KeypassXC muss die Passwortdatei stehts ausgetauscht werden aber sehr beliebt und hat eine Browserextention. Wer keinen Passwortmanager hat sollte sich die überlegen zu benutzten.
    \item \href{https://freeotp.github.io/}{FreeOTP} \& Alternativen\\
    Eine "-factor Authentifizierungssoftware, ein generell guter Schritt in Richtung Sicherheit seiner Dienste. Grundsätzlich sollten auch SMS-Verifizierungen vermieden werden und dies ist eine sehr gute Alternative.
    \item \href{https://de.libreoffice.org/download/download/}{LibreOffice}, \href{https://www.openoffice.org/de/download/}{OpenOffice} \& \href{https://proton.me/de/drive/docs}{ProtonDocs}\\
    Libre- und OpenOffice sind eine alternative zu Microsoft Office, jedoch ist diese für Studenten kostenlos verfügbar. ProtonDocs ist eine Alternative zu Google Workspace, jedoch gibt es bis jetzt nur Textdokumentsupport. 
    \item \href{https://code.visualstudio.com/download}{VS-Code}, \href{https://www.eclipse.org/downloads/}{Eclipse} \& \href{https://www.jetbrains.com/idea/}{IntelliJ IDEA}\\ Alle Editoren sind für Studenten frei zugänglich, IntelliJ's jedoch erst nach Aktivierung. Dies sind so mit die Empfohlenen Code-Editoren, es hängt von euch ab welchen ihr gerne benutzt.
    \item \href{https://education.github.com/discount_requests/application}{GitHub} \& \href{https://desktop.github.com/download/}{GitHub Desktop}\\
    GitHub ist eine der gängigen Versionierungs- und Code-Sharing-Platform, sie haben ein Studentenprogramm welchem einen Zugrif für viele weitere Softwares gibt. Dort ist unter anderem auch der KI-Codeeditor Copilot verfügbar. Die Desktop App bietet mit die beste und einsteigefreundlichste Variante um seine Datein mit GitHub zu synchronisieren und hat ein gutes Tutorial dazu.
    \item \href{https://www.blender.org/}{Blender} \& \href{https://manage.autodesk.com/products/maya}{Maya}\\
    Blender ist eine der gängigsten 3D-Software für die Erstellung von Modells, jedoch unterscheidet sie sich zum teil drastisch von den Industriestandarts. Maya von Autodesk st einder dieser, um diesen zu benutzten muss man sich als Student verifizieren.
    \item \href{https://krita.org/de/download/}{Krita}, \href{https://krita.org/de/download/}{Gimp}, \href{https://www.clipstudio.net/de/dl/}{Clip Studio Paint}\\
    Diese 3 Softwares sind ein Stück weit alternativen zu Photoshop im Bereich der Zeichenprogramme, die ersten beiden sind frei und Clip Studio kostet einmalig Geld um es zu benutzten.
    \item \href{https://inkscape.org/release/inkscape-1.3.2/windows/64-bit/msi/?redirected=1}{Inkscape}\\ Ist eine gute Alternative für Adobe Illustrator, er bietet viele möglichkeiten Vektorgrafiken zu erstellen und zu bearbeiten.
    \item \href{https://discord.com/download}{Discord}\\
    Discord ist eine Text- und Videochatsoftware, dies ist der standart bei Gamern und sonst eigentlich auch in der Hochschule Recht häufig verwendet. 
    \item \href{https://www.adobe.com/de/creativecloud/buy/students.html?promoid=65FN7X8B&mv=other}{Adobe}, \href{https://affinity.serif.com/de/}{Affinity} \& \href{https://www.magix.com/de/education/}{Magix}\\
    Diese 3 unternehmen bieten Softwarebundels im Kreativbereich an, welche auch viel genutzt werden. Die meisten haben eine Studentenoption, auch sind Magix und Affinity duch eine Einmalzahlung erwerbbar, wodurch sie auf lange dauer deutlich günstiger als Adobe sind. Hier sollte man sich jedoch selbst reinarbeiten, weil diese sind für das Studium \textbf{nicht} nötig.
\end{itemize}
