\subsection{Einstieg und Wiederhohlung}

\subsubsection{$\sin$ und $\cos$ im rechtwinkligem Dreieck}
Der $\sin$ ist das Verhältnis von der Gegenkathete zur Hypothenuse im rechtwinkligem Dreieck.\\~ \\
$\frac{Gegenkathete}{Hypothenuse}=\sin(\alpha)=\cos(\beta)$\\~\\
Der $\cos$ ist das Verhältnis der Ankathete im Vergleich zur Hypothenuse im rechtwinkligem Dreieck

\subsubsection{$\sin$ \& $\cos$ im Kreis}
\paragraph*{$\sin$ am Einheitskreis}~\\
Die Welle ist das Verhältnis des Sinus am Einheitskreis auf eine x-Achse übertragen.
\paragraph*{$\cos$ am Einheitskreis}~\\
Die Welle ist das Verhältnis des Cosinus am Einheitskreis auf eine x-Achse übertragen.\\~\\
\hyperlink{https://upload.wikimedia.org/wikipedia/commons/f/f3/Sinus_und_Cosinus_am_Einheitskreis.gif}{\textcolor{RedViolet}{\textbf{\fcolorbox{red}{white}{Darstellung von $\sin$ und $\cos$ im Einheitskreis}}}}\\
\begin{table}[h!]
\renewcommand{\arraystretch}{1.5}
\centering
\caption{Trigonometrische Werte für Winkel im Einheitskreis}
\begin{tabular}{c|c c c c c c c c c c c c c}
$x$ & $0$ & $\frac{\pi}{6}$ & $\frac{\pi}{4}$ & $\frac{\pi}{3}$ & $\frac{\pi}{2}$ & $\frac{2\pi}{3}$ & $\frac{3\pi}{4}$ & $\pi$ & $\frac{7\pi}{6}$ & $\frac{5\pi}{4}$ & $\frac{4\pi}{3}$ & $\frac{3\pi}{2}$ & $\frac{11\pi}{6}$ \\ \hline
$\sin x$ & $0$ & $\frac{1}{2}$ & $\frac{\sqrt{2}}{2}$ & $\frac{\sqrt{3}}{2}$ & $1$ & $\frac{\sqrt{3}}{2}$ & $\frac{\sqrt{2}}{2}$ & $0$ & $-\frac{1}{2}$ & $-\frac{\sqrt{2}}{2}$ & $-\frac{\sqrt{3}}{2}$ & $-1$ & $-\frac{1}{2}$ \\ 
$\cos x$ & $1$ & $\frac{\sqrt{3}}{2}$ & $\frac{\sqrt{2}}{2}$ & $\frac{1}{2}$ & $0$ & $-\frac{1}{2}$ & $-\frac{\sqrt{2}}{2}$ & $-1$ & $-\frac{\sqrt{3}}{2}$ & $-\frac{\sqrt{2}}{2}$ & $-\frac{1}{2}$ & $0$ & $\frac{1}{2}$ \\ 
$\tan x$ & $0$ & $\frac{1}{\sqrt{3}}$ & $1$ & $\sqrt{3}$ & $ / $ & $-\sqrt{3}$ & $-1$ & $0$ & $\frac{1}{\sqrt{3}}$ & $1$ & $\sqrt{3}$ & $ / $ & $-\frac{1}{\sqrt{3}}$ \\
\end{tabular}
\label{tab:trig}
\end{table}

\subsubsection{Additionstheoreme}
\begin{center}
    \includegraphics[width=1\textwidth]{Sem1/MINT/Bilder/Additionstheoreme_Trigonometrie.png}
\end{center}
$sin(x+y)=sin(x)*cos(y)+cos(x)*sin(y)$\\
$sin(30+120)=sin(30)*cos(120)+cos(30)*sin(120)$\\
$=\frac12*-\frac12+\frac{\sqrt3}{2}*\frac{\sqrt3}{2}$\\
$=-\frac14+\frac34=\underline{\frac12}$
\paragraph*{allgemeiner Fall}~\\
\begin{center}
    \includegraphics[width=100px]{Sem1/MINT/Bilder/download 1.png}
\end{center}
