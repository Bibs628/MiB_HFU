\section{SPO}

    Nachdem Studierende das Modul erfolgreich abgeschlossen haben, können sie:
    \begin{itemize}
        \item Wissen / Kenntnisse\\
            Geometrische und algebraische Fragestellungen präzise mithilfe der adäquaten Fachbegriffe artikulieren. 
            \newline
            Zentrale Grundbegriffe der Optik sicher wiedergeben.
        \item Verstehen\\
            Mathematische Sinnzusammenhänge und Beweiselemente bzw. Herleitungen erkennen verstehen und wiedergeben. 
            \newline
            Mathematische Modelle physikalischer Phänomene (z.B. geometrisch-optisches paraxiales Arbeitsmodell für abbildende Systeme) verstehen.
        \item Anwenden\\
            Techniken der Vektorrechnung und der Matrixalgebra auf geometrische Probleme anwenden. \newline
            Grundgesetze der Strahlenoptik auf einfache Kameraobjektivmodelle bzw. Fragestellungen der Fotografie anwenden.
        \item Analyse\\
            Geometrische Standardprobleme in der Ebene und im Raum analysieren. 
            \newline
            Angemessen ausgewählte physikalische Systeme und Strukturen selbstständig analysieren und beschreiben.
        \newpage
        \item Synthesis\\
            Für Frage- und Problemstellungen aus (Linearer) Algebra und Geometrie unter den bereitgestellten Hilfsmitteln die jeweils adäquaten auswählen. 
            \newline
            Ein geeignetes eingegrenztes, für die Medientechnik relevantes Thema aus Optik oder Akustik im Überblick darstellen.
        \item Evaluation\\
            Verschiedene Verfahren (z.B. zur Bestimmung affiner Transformationen) hinsichtlich Übersichtlichkeit und Aufwand abwägen.
    \end{itemize}
    
\newpage