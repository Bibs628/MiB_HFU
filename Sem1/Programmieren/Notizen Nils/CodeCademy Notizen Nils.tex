\section{CodeCademy Notizen}

Hi, Nils Hier. Ich bin mirt nicht sicher wie ich den Inhalt vom Herrn Eisenbiegler gut weitergeben kann, vorallem da ich hier auch gestruggelt habe. Darum habe ich hier Angefangen meinen Vortschritt des CodeCademy Java Kursen in Notizen zu fassen um es Inhaltlich sinnvoll wiederzugeben. Zusätzlich habe ich den Abscnitt hier noch einmal von Daniel rüber schauen lassen, da er zumindest mehr Verständis von Code hat als ich.

\subsection{Lesson 1 - Hello Word.}
\paragraph{Java's entstehung}~\\
Zu aller erst einmal etwas zu der Programmiersprache, Java wurde 1995 von Sun Microsystems veröffentlicht. Java ist dafür bekannt einfach, sicher, robust und portierbar zu sein. Dies macht es heute immer noch zu einer der beliebtesten Programmiersprachen bis heute.

\paragraph{Java}~\\
Was Java so gut macht ist unter anderem ihre VM (Virtual Machiene), die es erlaubt Java auf eigentlich jeder Plattform zu laufen.\\
Programmiersprachen sind mit Hilfe der \textit{Syntaxe} aufgebaut. Wir schreiben diese in Datein, welche dann von dem PC die einprogrammierte Aufgabe ausführt. 

\paragraph{Hello World - jetzt aber Wirklich}~\\
\begin{verbatim}
public class HelloWorld {
  public static void main(String[] args) {
    System.out.println("Hello World!");
  }
}
\end{verbatim}

Was hier zu beachten ist das die \textit{class} sozusagen das Gebiet bezeichnet, (wie eine Adresse) und alles in dieser Adresse ne Fabrik ist die eine oder viele Aufgaben erfüllt.\\
Wichtig hier ist jedoch das wenn die Datei \glqq Hello World \grqq heißt, muss entsprechend auch die Klasse \glqq Hello World \grqq heißen. Der Name der Klasse ist immer der gleiche wie der der Datei. 
\subsection{Lesson 2 - Java Datein}
Java Datein enden immer mit der endung \glqq .java \grqq
In dieser Datei steht eine Klasse, im letzten Beispiel: 

\begin{verbatim}
public class HelloWorld {
  
}
\end{verbatim}
Santaxe in den geschweiften Klammern \glqq \{ \}\grqq~gehören zu der zugehörigen Klasse.\\
Jede Datei hat eine Klasse die nach der Datei benannt ist, dies ist dann die Hauptklasse. Die Bennenungskonvention ist in der regel so, das jedes neue Wort groß geschrieben wird um es verständlich zu machen. So heißt dann die Datei: \glqq HelloWold.java\grqq~und die Klasse: \glqq HelloWolrd\grqq.\\~\\
In unserer Klasse hatten wir eine main()-Methode, welche unsere Anwendungsaufganen aufliestet:
\begin{verbatim}
public static void main(String[] args) {

}
\end{verbatim}

Dieses Programm hatte die Aufgabe \glqq Hello World \glqq~ auszugeben, was durch den folgenden Befehl erreicht wurde: 
\begin{verbatim}
    System.out.println("Hello World");
\end{verbatim}
\subsection{Lesson 3 - Print statements}
Hier schauen wir etwas tiefer in die Funktionsweise des Print-Statements rein.
\begin{verbatim}
    System.out.println("Hello World");
\end{verbatim}
\begin{itemize}
    \item System\\
    ist in Java eingebaut und bietet nützliche Tools für unser Programm
    \item out\\
    ist die abkürzung von Output
    \item println\\
    steht für "printline"
\end{itemize}
Wir können \glqq System.out.println() \grqq~immer verwenden um
\subsection{Lesson x - x}

\subsection{Lesson x - x}

\subsection{Lesson x - x}

\subsection{Lesson x - x}

\subsection{Lesson x - x}

\subsection{Lesson x - x}

\subsection{Lesson x - x}

\subsection{Lesson x - x}

\subsection{Lesson x - x}

\subsection{Lesson x - x}

\subsection{Lesson x - x}

\subsection{Lesson x - x}

\subsection{Lesson x - x}

\subsection{Lesson x - x}

\subsection{Lesson x - x}



Java Glossar:

\begin{itemize}
    \item Class
    \item 
    \item 
    \item 
    \item 
    \item 
    \item 
    \item 
    \item 
    \item 
    \item 
    \item 
    \item 
    \item 
    \item 
    
\end{itemize}