\subsection{Schallwellen}


    \subsubsection{Frequenz}
                \paragraph{Definition} 
                    Die Frequenz ist in Physik und Technik ein Maß dafür, wie schnell bei einem periodischen Vorgang die Wiederholungen aufeinander folgen, z. B. bei einer fortdauernden Schwingung. Die Frequenz ist der Kehrwert der Periodendauer.\\Die Einheit der Frequenz ist die abgeleitete SI-Einheit mit dem besonderen Namen Hertz (Einheitenzeichen  Hz); $1 Hz = 1 s-1$ („eins pro Sekunde“). \\~\\~\\
    $f=\frac{\Delta N}{\Delta t}$\\
    $Frequenz=\frac{Anzahl~Wiederholungen}{Zeit}$
    \\~\\

    
    \subsubsection{Periodendauer}
        Dauer einer Schwingung in Sekunden, T = 1/f



    \subsubsection{Amplitude}
        T = Perioden, t= Zeit, Amplitude gibt die Lautstärke an


    \subsubsection{Wellenlänge}
    Die Wellenlänge ist der Abstand zwischen den Wellenbergen, Je Höher die Frequenz, desto kürzer die Wellenlänge\\
 $c=\lambda*f$\\

Ab knapp 20 Wiederholungen werden Klänge und Bilder zu einem einzigen Ton bzw. Video.

\begin{table}[h]
\begin{center}
\begin{tabular}{l|l}
\hline
\rowcolor{YellowGreen!50!} \multicolumn{1}{|l|}{Frequenz} & \multicolumn{1}{l|}{Wellenlänge $\lambda$} \\ \hline
16Hz                           & 21,2m                                      \\
20Hz                           & 17m                                        \\
100Hz                          & 3,4m                                       \\
1.000Hz                        & 0,34m                                      \\
10.000Hz                       & 0,034m                                     \\
16.000Hz                       & 0,021m                                     \\
20.000Hz                       & 0,017m                                    
\end{tabular}
\end{center}
\end{table}
\newpage

    \subsubsection{Oktaven}
    Eine Oktave höher Bedeutet eine verdopplung der [[Frequenz]], dies bedeutet aus 125 Hz werden 250 Hz, 500Hz, 1000Hz, 2000Hz, ...
Dies bedeutet das der Abstand für unsere Ohren der Abstand gleich groß zwischen diesen werten ist und um es entsprechend für unsere Ohren darzustellen werden diese alle Logarythmisch aufgetragen. \\
$20~Hz = \frac{1}{20}Sekunde=50ms$


    \subsubsection{Schallausbreitung}
        Der Tonimpuls wird von Molekül auf Molekül weitergetragen, wie eine Welle im Wasser.  Ein Molekül "schubst"/gibt seine Energie an andere weiter. Es schwingt trotzdem nur an seiner Stelle\\~\\
Der Schall ist eine Wellenbewegung, dieser breitet sich vom Ursprung aus in alle Richtungen aus. Stimmen sind Direktionale Töne, vor dem Kopf besser als hinter dem Kopf zu hören. \\~\\
Schallgeschwindigkeit ist für alle Frequenzen gleich schnell, es ändert sich ein wenig bei Temperaturen und nach Stoff wo sich der Schall bewegt.
Eisen z.B. 5km / Sekunde

    \paragraph*{Schallausbreitung}~\\
pro m 3ms = pro km 3sek\\
c bei 0°C: 331,5m/s\\
c bei 20°C: 343m/s


    \subsubsection{Schallbeugung}
        Töne mit langer Wellenlänge beugen sich besser um Hindernisse, weil sie aufgrund ihrem Abstand der Wellen besser um Strukturen herum kommen. 


    \subsubsection{Echogrenze}
        50m/s = 1/20tel Sekunde\\
entspricht 20 Impulsen pro Sekunde = 20Hz


    \subsubsection{Residualeffekt}
    \begin{itemize}
        \item Ohr schliest aus Obertonstruktur auf (fehlenden) Grundton = Psychoakkustik
        \item Oberwellen addieren sich zur Grundwelle auf (Überlagerung von Wellen)
    \end{itemize}


    \subsubsection{Logarithmische Tonwahrnehmung}
        10 Oktaven sind wahrnehmbar.
