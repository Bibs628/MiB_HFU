\subsection{Lokalisation}
Lokalisation ist die Richtungswahrnehmung einer Schallquelle\\~\\
Nicht zu verwechseln mit der Räumlichkeit: Das ist der wahrgenommene Raum, in dem eine Schallquelle zu hören ist.


\notebox{Wenn ihr mehr zu Lokalisation und anderen Tonereignissen erfahren wollt, der YouTube Kanal \href{https://www.youtube.com/@veritasium}{Veritasium} hat zu Auudioeffekten ein \href{https://youtu.be/Sn07AMCfaAI?si=0VljFCL331ypbOom}{Video} veröffentlicht. Dort sind einige der von Herr Reusch gezeigten Phänomene anschaulich dargestellt.}

\subsubsection{Horizontale Lokalisation}
durch interaurale Laufzeitunterschiede und frequenzabhängige Pegelunterschiede. Wir drehen unbewusst den Kopf ständig ein wenig, um genauer zu lokalisieren.\\

\textbf{Lokalisierung auf der Horizontalebene möglich durch:}\\
\paragraph{Interaurale laufzeitdifferenz}~\\
durch 17cm Ohrenabstand max. 0,63 ms. Geringste wahrnehmbare Differenz liegt bei 0,03 ms, entsprechend 3°-5° aus der Mitte, entsprechend 1 cm Laufzeitunterschied.
\paragraph{Pegeldifferenzen}~\\
zwischen linkem und rechtem Ohr: Unterhalb 300Hz keine Unterschiede aufgrund Beugung um den Kopf. Oberhalb gibt es Pegelunterschiede, also Spektraldifferenzen, die zur Ortung führen. Ungenauer als Laufzeitunterschiede!\\~\\
Ohr wertet im Bereich oberhalb 1600 Hz Laufzeit und Pegel aus und ist sehr präzise. Unterhalb von 1600Hz wird vorwiegend die Laufzeit ausgewertet.

\subsubsection{Vertikale Lokalisation}
nur durch durch spektrale Veränderungen, Einfluss der Ohrmuscheln: Richtungsbestimmende (Frequenz)-Bänder nach Blauert:

Und: unser Ohr kann aus den Reflexionen des Raumes ebenfalls auf den Standort der Quelle schliessen.