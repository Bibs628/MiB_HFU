\chapter{Programmieren}

Dieser Kurs ist \textbf{NICHT} für komplette Programmieranfänger ohne großen Zeitaufwand zu bewältigen. Der Eisenbiegler ist ein geborener Informatiker, jedoch leider nicht ein guter Professor. Ich empfehle euch Java unabhängig von seinen Resourcen zu lernen, seine Übungsaufgaben sind jedoch Sinnvoll konzipiert. Es steigen regelmäßig über mehrere Semester mehrere Personen, meist ohne Vorkentnisse, an der Bildverarbeitungsaufgabe aus.\\
Wenn ihr jedoch schon Vorkentnisse habt kann euch der Herr Eisenbiegler gut weiterhelfen um entsprechend eure Kenntnisse zu vertiefen und euch Sinnvolle Strukturen an die Hand geben.\\
Ich persönlich bin halt auch zwischendurch Ausgestiegen und er konnte mir persönlich nicht die Hilfe bieten die ich gebraucht habe. :/ \\
% nicht sehr motivierend bro
Wer einen Onlinekurs nebenbei belegen will empfehle ich Kurse von \href{https://open.hpi.de/courses}{openhpi}, dies sind kostenfreie Kurse. Speziell empfehle ich diesen \href{https://open.hpi.de/courses/javaeinstieg2020}{Java Kurs}, als alternative zu diesem gibt es auch noch die  \href{https://open.hpi.de/courses/javaeinstieg-schule2024}{Schülerversion}. Dieser wurde mir jedoch von einer Professorin aus der Bekanntschaft empfohlen. \\ Ein weiterer Anbieter von solchen Kursen ist \href{https://www.codecademy.com/learn/learn-java}{CodeCademy}, diese beiten unter anderem auch kostenlose urse zu \href{https://www.codecademy.com/learn/learn-java}{Java} oder weiteren wie \href{https://www.codecademy.com/learn/learn-python-3}{Python 3}.
\\~\\
\emph{- Nils}

\newpage
\section{SPO}

    Nachdem Studierende das Modul erfolgreich abgeschlossen haben, können sie:
    \begin{itemize}
        \item Wissen / Kenntnisse\\
            Die Sprachelemente einer imperativen Programmiersprache benennen.
        \item Verstehen\\
            Die Bedeutung eines imperativen Computerprogramms erklären.
        \item Anwenden\\
            Mit einer integrierten Entwicklungsumgebung arbeiten.
        \item Analyse\\
            Den Ablauf eines vorgegebenen imperativen Computerprogramms beschreiben.
        \item Synthesis\\
            Zu einer einfachen Aufgabenstellung ein imperatives Computerprogramm selbstständig implementieren.
        \item Evaluation\\
            Unterschiedliche Computerprogramme in Bezug auf ihre Effizienz miteinander vergleichen.
    \end{itemize}
    
\newpage
\subsection{Theorie}\newpage
\subsection{Übungsaufgaben}

\input{Sem1/Programmieren/Uebungen_Eisenbiegler/A1P_Einführung}
\input{Sem1/Programmieren/Uebungen_Eisenbiegler/A2P_ImperativeProg}
\input{Sem1/Programmieren/Uebungen_Eisenbiegler/A3P_ImperativeProg}
\input{Sem1/Programmieren/Uebungen_Eisenbiegler/A4P_Bildverarbeitung}
\input{Sem1/Programmieren/Uebungen_Eisenbiegler/A5P_ImperativeProg}
\input{Sem1/Programmieren/Uebungen_Eisenbiegler/A6P_Rekursion}
\input{Sem1/Programmieren/Uebungen_Eisenbiegler/A7P_Objekte}
\input{Sem1/Programmieren/Uebungen_Eisenbiegler/A8P_VerketteteObjekte}
\input{Sem1/Programmieren/Uebungen_Eisenbiegler/A9P_ExceptionHandle}
\input{Sem1/Programmieren/Uebungen_Eisenbiegler/A10P_ExceptionsAbst}
\input{Sem1/Programmieren/Uebungen_Eisenbiegler/A11P_PersonenImKrankenhaus}
\input{Sem1/Programmieren/Uebungen_Eisenbiegler/A12P_Fensterbestellung}